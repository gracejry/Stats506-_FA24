% Options for packages loaded elsewhere
\PassOptionsToPackage{unicode}{hyperref}
\PassOptionsToPackage{hyphens}{url}
\PassOptionsToPackage{dvipsnames,svgnames,x11names}{xcolor}
%
\documentclass[
  letterpaper,
  DIV=11,
  numbers=noendperiod]{scrartcl}

\usepackage{amsmath,amssymb}
\usepackage{iftex}
\ifPDFTeX
  \usepackage[T1]{fontenc}
  \usepackage[utf8]{inputenc}
  \usepackage{textcomp} % provide euro and other symbols
\else % if luatex or xetex
  \usepackage{unicode-math}
  \defaultfontfeatures{Scale=MatchLowercase}
  \defaultfontfeatures[\rmfamily]{Ligatures=TeX,Scale=1}
\fi
\usepackage{lmodern}
\ifPDFTeX\else  
    % xetex/luatex font selection
\fi
% Use upquote if available, for straight quotes in verbatim environments
\IfFileExists{upquote.sty}{\usepackage{upquote}}{}
\IfFileExists{microtype.sty}{% use microtype if available
  \usepackage[]{microtype}
  \UseMicrotypeSet[protrusion]{basicmath} % disable protrusion for tt fonts
}{}
\makeatletter
\@ifundefined{KOMAClassName}{% if non-KOMA class
  \IfFileExists{parskip.sty}{%
    \usepackage{parskip}
  }{% else
    \setlength{\parindent}{0pt}
    \setlength{\parskip}{6pt plus 2pt minus 1pt}}
}{% if KOMA class
  \KOMAoptions{parskip=half}}
\makeatother
\usepackage{xcolor}
\setlength{\emergencystretch}{3em} % prevent overfull lines
\setcounter{secnumdepth}{-\maxdimen} % remove section numbering
% Make \paragraph and \subparagraph free-standing
\ifx\paragraph\undefined\else
  \let\oldparagraph\paragraph
  \renewcommand{\paragraph}[1]{\oldparagraph{#1}\mbox{}}
\fi
\ifx\subparagraph\undefined\else
  \let\oldsubparagraph\subparagraph
  \renewcommand{\subparagraph}[1]{\oldsubparagraph{#1}\mbox{}}
\fi

\usepackage{color}
\usepackage{fancyvrb}
\newcommand{\VerbBar}{|}
\newcommand{\VERB}{\Verb[commandchars=\\\{\}]}
\DefineVerbatimEnvironment{Highlighting}{Verbatim}{commandchars=\\\{\}}
% Add ',fontsize=\small' for more characters per line
\usepackage{framed}
\definecolor{shadecolor}{RGB}{241,243,245}
\newenvironment{Shaded}{\begin{snugshade}}{\end{snugshade}}
\newcommand{\AlertTok}[1]{\textcolor[rgb]{0.68,0.00,0.00}{#1}}
\newcommand{\AnnotationTok}[1]{\textcolor[rgb]{0.37,0.37,0.37}{#1}}
\newcommand{\AttributeTok}[1]{\textcolor[rgb]{0.40,0.45,0.13}{#1}}
\newcommand{\BaseNTok}[1]{\textcolor[rgb]{0.68,0.00,0.00}{#1}}
\newcommand{\BuiltInTok}[1]{\textcolor[rgb]{0.00,0.23,0.31}{#1}}
\newcommand{\CharTok}[1]{\textcolor[rgb]{0.13,0.47,0.30}{#1}}
\newcommand{\CommentTok}[1]{\textcolor[rgb]{0.37,0.37,0.37}{#1}}
\newcommand{\CommentVarTok}[1]{\textcolor[rgb]{0.37,0.37,0.37}{\textit{#1}}}
\newcommand{\ConstantTok}[1]{\textcolor[rgb]{0.56,0.35,0.01}{#1}}
\newcommand{\ControlFlowTok}[1]{\textcolor[rgb]{0.00,0.23,0.31}{#1}}
\newcommand{\DataTypeTok}[1]{\textcolor[rgb]{0.68,0.00,0.00}{#1}}
\newcommand{\DecValTok}[1]{\textcolor[rgb]{0.68,0.00,0.00}{#1}}
\newcommand{\DocumentationTok}[1]{\textcolor[rgb]{0.37,0.37,0.37}{\textit{#1}}}
\newcommand{\ErrorTok}[1]{\textcolor[rgb]{0.68,0.00,0.00}{#1}}
\newcommand{\ExtensionTok}[1]{\textcolor[rgb]{0.00,0.23,0.31}{#1}}
\newcommand{\FloatTok}[1]{\textcolor[rgb]{0.68,0.00,0.00}{#1}}
\newcommand{\FunctionTok}[1]{\textcolor[rgb]{0.28,0.35,0.67}{#1}}
\newcommand{\ImportTok}[1]{\textcolor[rgb]{0.00,0.46,0.62}{#1}}
\newcommand{\InformationTok}[1]{\textcolor[rgb]{0.37,0.37,0.37}{#1}}
\newcommand{\KeywordTok}[1]{\textcolor[rgb]{0.00,0.23,0.31}{#1}}
\newcommand{\NormalTok}[1]{\textcolor[rgb]{0.00,0.23,0.31}{#1}}
\newcommand{\OperatorTok}[1]{\textcolor[rgb]{0.37,0.37,0.37}{#1}}
\newcommand{\OtherTok}[1]{\textcolor[rgb]{0.00,0.23,0.31}{#1}}
\newcommand{\PreprocessorTok}[1]{\textcolor[rgb]{0.68,0.00,0.00}{#1}}
\newcommand{\RegionMarkerTok}[1]{\textcolor[rgb]{0.00,0.23,0.31}{#1}}
\newcommand{\SpecialCharTok}[1]{\textcolor[rgb]{0.37,0.37,0.37}{#1}}
\newcommand{\SpecialStringTok}[1]{\textcolor[rgb]{0.13,0.47,0.30}{#1}}
\newcommand{\StringTok}[1]{\textcolor[rgb]{0.13,0.47,0.30}{#1}}
\newcommand{\VariableTok}[1]{\textcolor[rgb]{0.07,0.07,0.07}{#1}}
\newcommand{\VerbatimStringTok}[1]{\textcolor[rgb]{0.13,0.47,0.30}{#1}}
\newcommand{\WarningTok}[1]{\textcolor[rgb]{0.37,0.37,0.37}{\textit{#1}}}

\providecommand{\tightlist}{%
  \setlength{\itemsep}{0pt}\setlength{\parskip}{0pt}}\usepackage{longtable,booktabs,array}
\usepackage{calc} % for calculating minipage widths
% Correct order of tables after \paragraph or \subparagraph
\usepackage{etoolbox}
\makeatletter
\patchcmd\longtable{\par}{\if@noskipsec\mbox{}\fi\par}{}{}
\makeatother
% Allow footnotes in longtable head/foot
\IfFileExists{footnotehyper.sty}{\usepackage{footnotehyper}}{\usepackage{footnote}}
\makesavenoteenv{longtable}
\usepackage{graphicx}
\makeatletter
\def\maxwidth{\ifdim\Gin@nat@width>\linewidth\linewidth\else\Gin@nat@width\fi}
\def\maxheight{\ifdim\Gin@nat@height>\textheight\textheight\else\Gin@nat@height\fi}
\makeatother
% Scale images if necessary, so that they will not overflow the page
% margins by default, and it is still possible to overwrite the defaults
% using explicit options in \includegraphics[width, height, ...]{}
\setkeys{Gin}{width=\maxwidth,height=\maxheight,keepaspectratio}
% Set default figure placement to htbp
\makeatletter
\def\fps@figure{htbp}
\makeatother

\KOMAoption{captions}{tableheading}
\makeatletter
\@ifpackageloaded{caption}{}{\usepackage{caption}}
\AtBeginDocument{%
\ifdefined\contentsname
  \renewcommand*\contentsname{Table of contents}
\else
  \newcommand\contentsname{Table of contents}
\fi
\ifdefined\listfigurename
  \renewcommand*\listfigurename{List of Figures}
\else
  \newcommand\listfigurename{List of Figures}
\fi
\ifdefined\listtablename
  \renewcommand*\listtablename{List of Tables}
\else
  \newcommand\listtablename{List of Tables}
\fi
\ifdefined\figurename
  \renewcommand*\figurename{Figure}
\else
  \newcommand\figurename{Figure}
\fi
\ifdefined\tablename
  \renewcommand*\tablename{Table}
\else
  \newcommand\tablename{Table}
\fi
}
\@ifpackageloaded{float}{}{\usepackage{float}}
\floatstyle{ruled}
\@ifundefined{c@chapter}{\newfloat{codelisting}{h}{lop}}{\newfloat{codelisting}{h}{lop}[chapter]}
\floatname{codelisting}{Listing}
\newcommand*\listoflistings{\listof{codelisting}{List of Listings}}
\makeatother
\makeatletter
\makeatother
\makeatletter
\@ifpackageloaded{caption}{}{\usepackage{caption}}
\@ifpackageloaded{subcaption}{}{\usepackage{subcaption}}
\makeatother
\ifLuaTeX
  \usepackage{selnolig}  % disable illegal ligatures
\fi
\usepackage{bookmark}

\IfFileExists{xurl.sty}{\usepackage{xurl}}{} % add URL line breaks if available
\urlstyle{same} % disable monospaced font for URLs
\hypersetup{
  pdftitle={506a1},
  pdfauthor={Ruyue Jiang},
  colorlinks=true,
  linkcolor={blue},
  filecolor={Maroon},
  citecolor={Blue},
  urlcolor={Blue},
  pdfcreator={LaTeX via pandoc}}

\title{506a1}
\author{Ruyue Jiang}
\date{}

\begin{document}
\maketitle

\subsection{Quarto}\label{quarto}

Quarto enables you to weave together content and executable code into a
finished document. To learn more about Quarto see
\url{https://quarto.org}.

\begin{Shaded}
\begin{Highlighting}[]
\FunctionTok{library}\NormalTok{(tidyverse)}
\end{Highlighting}
\end{Shaded}

\begin{verbatim}
-- Attaching core tidyverse packages ------------------------ tidyverse 2.0.0 --
v dplyr     1.1.4     v readr     2.1.5
v forcats   1.0.0     v stringr   1.5.1
v ggplot2   3.5.1     v tibble    3.2.1
v lubridate 1.9.3     v tidyr     1.3.1
v purrr     1.0.2     
-- Conflicts ------------------------------------------ tidyverse_conflicts() --
x dplyr::filter() masks stats::filter()
x dplyr::lag()    masks stats::lag()
i Use the conflicted package (<http://conflicted.r-lib.org/>) to force all conflicts to become errors
\end{verbatim}

\subsection{Problem 1}\label{problem-1}

From \url{https://archive.ics.uci.edu/dataset/109/wine}, download the
data set about wine. It contains two files of interest - ``wine.data''
with the actual rectangular data set, and ``wine.names'' with some
information about the data. (Both files are plain-text - you can open
then in any text editor, including directly in RStudio.)

\begin{enumerate}
\def\labelenumi{\alph{enumi}.}
\tightlist
\item
  Import the data into a \texttt{data.frame} in R. Use the information
  in the ``wine.names'' file to give appropriate column names. (Note:
  Downloading and unzipping the file can take place outside of your
  submitted document, but importing the file should be in the
  submission.)
\end{enumerate}

\begin{Shaded}
\begin{Highlighting}[]
\FunctionTok{setwd}\NormalTok{(}\StringTok{"/Users/gracejiang/Downloads/wine"}\NormalTok{)}
\NormalTok{wine\_data }\OtherTok{\textless{}{-}} \FunctionTok{read.table}\NormalTok{(}\StringTok{"wine.data"}\NormalTok{, }\AttributeTok{sep =} \StringTok{","}\NormalTok{, }\AttributeTok{header =} \ConstantTok{FALSE}\NormalTok{)}
\FunctionTok{colnames}\NormalTok{(wine\_data) }\OtherTok{\textless{}{-}} \FunctionTok{c}\NormalTok{(}\StringTok{"Class"}\NormalTok{, }\StringTok{"Alcohol"}\NormalTok{, }\StringTok{"Malic acid"}\NormalTok{, }\StringTok{"Ash"}\NormalTok{, }\StringTok{"Alcalinity of ash"}\NormalTok{, }\StringTok{"Magnesium"}\NormalTok{, }\StringTok{"Total phenols"}\NormalTok{, }\StringTok{"Flavanoids"}\NormalTok{, }\StringTok{"Nonflavanoid phenols"}\NormalTok{, }\StringTok{"Proanthocyanins"}\NormalTok{, }\StringTok{"Color intensity"}\NormalTok{, }\StringTok{"Hue"}\NormalTok{, }\StringTok{"OD280/OD315 of diluted wines"}\NormalTok{, }\StringTok{"Proline"}\NormalTok{)}
\FunctionTok{head}\NormalTok{(wine\_data)}
\end{Highlighting}
\end{Shaded}

\begin{verbatim}
  Class Alcohol Malic acid  Ash Alcalinity of ash Magnesium Total phenols
1     1   14.23       1.71 2.43              15.6       127          2.80
2     1   13.20       1.78 2.14              11.2       100          2.65
3     1   13.16       2.36 2.67              18.6       101          2.80
4     1   14.37       1.95 2.50              16.8       113          3.85
5     1   13.24       2.59 2.87              21.0       118          2.80
6     1   14.20       1.76 2.45              15.2       112          3.27
  Flavanoids Nonflavanoid phenols Proanthocyanins Color intensity  Hue
1       3.06                 0.28            2.29            5.64 1.04
2       2.76                 0.26            1.28            4.38 1.05
3       3.24                 0.30            2.81            5.68 1.03
4       3.49                 0.24            2.18            7.80 0.86
5       2.69                 0.39            1.82            4.32 1.04
6       3.39                 0.34            1.97            6.75 1.05
  OD280/OD315 of diluted wines Proline
1                         3.92    1065
2                         3.40    1050
3                         3.17    1185
4                         3.45    1480
5                         2.93     735
6                         2.85    1450
\end{verbatim}

\begin{enumerate}
\def\labelenumi{\alph{enumi}.}
\setcounter{enumi}{1}
\tightlist
\item
  The data contains information on three different classes of wine.
  Check and report that the number of wines within each class is correct
  as reported in ``wine.names''.
\end{enumerate}

\begin{Shaded}
\begin{Highlighting}[]
\NormalTok{wine\_number }\OtherTok{\textless{}{-}}\NormalTok{ wine\_data }\SpecialCharTok{\%\textgreater{}\%} \FunctionTok{group\_by}\NormalTok{(Class) }\SpecialCharTok{\%\textgreater{}\%} \FunctionTok{count}\NormalTok{()}
\NormalTok{wine\_number}
\end{Highlighting}
\end{Shaded}

\begin{verbatim}
# A tibble: 3 x 2
# Groups:   Class [3]
  Class     n
  <int> <int>
1     1    59
2     2    71
3     3    48
\end{verbatim}

The number of wines within each class is correct as reported in
``wine,names''. There are 59 in Class 1, 71 in Class 2 and 48 in Class
3.

\begin{enumerate}
\def\labelenumi{\alph{enumi}.}
\setcounter{enumi}{2}
\item
  Use the data to answer the following questions:

  \begin{enumerate}
  \def\labelenumii{\arabic{enumii}.}
  \item
    What is the correlation between alcohol content and color intensity?
  \item
    Which class has the highest correlation? Which has the lowest?
  \item
    What is the alcohol content of the wine with the highest color
    intensity?
  \item
    What percentage of wines had a higher content of proanthocyanins
    compare to ash?
  \end{enumerate}
\end{enumerate}

\begin{Shaded}
\begin{Highlighting}[]
\NormalTok{corr1 }\OtherTok{\textless{}{-}} \FunctionTok{cor}\NormalTok{(wine\_data}\SpecialCharTok{$}\NormalTok{Alcohol, wine\_data}\SpecialCharTok{$}\StringTok{\textasciigrave{}}\AttributeTok{Color intensity}\StringTok{\textasciigrave{}}\NormalTok{)}
\NormalTok{corr1}
\end{Highlighting}
\end{Shaded}

\begin{verbatim}
[1] 0.5463642
\end{verbatim}

\begin{Shaded}
\begin{Highlighting}[]
\NormalTok{corr\_class }\OtherTok{\textless{}{-}}\NormalTok{ wine\_data }\SpecialCharTok{\%\textgreater{}\%} 
                  \FunctionTok{group\_by}\NormalTok{(Class) }\SpecialCharTok{\%\textgreater{}\%} 
                  \FunctionTok{summarize}\NormalTok{(}\FunctionTok{cor}\NormalTok{(Alcohol,}\StringTok{\textasciigrave{}}\AttributeTok{Color intensity}\StringTok{\textasciigrave{}}\NormalTok{))}
\NormalTok{corr\_class}
\end{Highlighting}
\end{Shaded}

\begin{verbatim}
# A tibble: 3 x 2
  Class `cor(Alcohol, \`Color intensity\`)`
  <int>                               <dbl>
1     1                               0.408
2     2                               0.270
3     3                               0.350
\end{verbatim}

\begin{Shaded}
\begin{Highlighting}[]
\NormalTok{highest\_value }\OtherTok{\textless{}{-}}\NormalTok{ wine\_data }\SpecialCharTok{\%\textgreater{}\%} 
                 \FunctionTok{select}\NormalTok{(Alcohol, }\StringTok{\textasciigrave{}}\AttributeTok{Color intensity}\StringTok{\textasciigrave{}}\NormalTok{) }\SpecialCharTok{\%\textgreater{}\%} 
                 \FunctionTok{arrange}\NormalTok{(}\FunctionTok{desc}\NormalTok{(}\StringTok{\textasciigrave{}}\AttributeTok{Color intensity}\StringTok{\textasciigrave{}}\NormalTok{)) }\SpecialCharTok{\%\textgreater{}\%} \FunctionTok{head}\NormalTok{(}\DecValTok{1}\NormalTok{)}
\NormalTok{highest\_value}
\end{Highlighting}
\end{Shaded}

\begin{verbatim}
  Alcohol Color intensity
1   14.34              13
\end{verbatim}

\begin{Shaded}
\begin{Highlighting}[]
\NormalTok{wine\_percent }\OtherTok{\textless{}{-}}\NormalTok{ wine\_data }\SpecialCharTok{\%\textgreater{}\%} \FunctionTok{summarize}\NormalTok{(}\FunctionTok{mean}\NormalTok{(Proanthocyanins }\SpecialCharTok{\textgreater{}}\NormalTok{ Ash, }\AttributeTok{na.rm =} \ConstantTok{TRUE}\NormalTok{))}
\NormalTok{wine\_percent}
\end{Highlighting}
\end{Shaded}

\begin{verbatim}
  mean(Proanthocyanins > Ash, na.rm = TRUE)
1                                0.08426966
\end{verbatim}

\begin{enumerate}
\def\labelenumi{\alph{enumi}.}
\setcounter{enumi}{3}
\tightlist
\item
  Create a table identifying the average value of each variable,
  providing one row for the overall average, and one row per class with
  class averages. (This table does not need to be ``fancy'' but should
  clearly identify what each value represents.)
\end{enumerate}

\begin{Shaded}
\begin{Highlighting}[]
\NormalTok{mean\_all }\OtherTok{\textless{}{-}}\NormalTok{ wine\_data }\SpecialCharTok{\%\textgreater{}\%} 
            \FunctionTok{summarise\_all}\NormalTok{(mean) }\SpecialCharTok{\%\textgreater{}\%} 
            \FunctionTok{mutate}\NormalTok{(}\AttributeTok{Class =} \StringTok{"Overall"}\NormalTok{)}

\NormalTok{mean\_by\_class }\OtherTok{\textless{}{-}}\NormalTok{ wine\_data }\SpecialCharTok{\%\textgreater{}\%} 
                 \FunctionTok{group\_by}\NormalTok{(Class) }\SpecialCharTok{\%\textgreater{}\%} 
                 \FunctionTok{summarize\_all}\NormalTok{(mean) }\SpecialCharTok{\%\textgreater{}\%} 
                 \FunctionTok{mutate}\NormalTok{(}\AttributeTok{Class =} \FunctionTok{as.character}\NormalTok{(Class))}
\NormalTok{table1 }\OtherTok{\textless{}{-}}  \FunctionTok{bind\_rows}\NormalTok{(mean\_all, mean\_by\_class)}
\NormalTok{table1}
\end{Highlighting}
\end{Shaded}

\begin{verbatim}
    Class  Alcohol Malic acid      Ash Alcalinity of ash Magnesium
1 Overall 13.00062   2.336348 2.366517          19.49494  99.74157
2       1 13.74475   2.010678 2.455593          17.03729 106.33898
3       2 12.27873   1.932676 2.244789          20.23803  94.54930
4       3 13.15375   3.333750 2.437083          21.41667  99.31250
  Total phenols Flavanoids Nonflavanoid phenols Proanthocyanins Color intensity
1      2.295112  2.0292697            0.3618539        1.590899        5.058090
2      2.840169  2.9823729            0.2900000        1.899322        5.528305
3      2.258873  2.0808451            0.3636620        1.630282        3.086620
4      1.678750  0.7814583            0.4475000        1.153542        7.396250
        Hue OD280/OD315 of diluted wines   Proline
1 0.9574494                     2.611685  746.8933
2 1.0620339                     3.157797 1115.7119
3 1.0562817                     2.785352  519.5070
4 0.6827083                     1.683542  629.8958
\end{verbatim}

The first row of the table shows the average value of each variable. And
the second to the forth row each shows the average value of each class
of each variable.

\begin{enumerate}
\def\labelenumi{\alph{enumi}.}
\setcounter{enumi}{4}
\tightlist
\item
  Carry out a series of t-tests to examine whether the level of phenols
  differs across the three classes. Present the R output and interpret
  the results. (You may use an existing R function to carry out the
  t-test, or for \textbf{minor extra credit}, manually write your own
  calculation of the t-test p-values.)
\end{enumerate}

\begin{Shaded}
\begin{Highlighting}[]
\NormalTok{class1 }\OtherTok{\textless{}{-}}\NormalTok{ wine\_data[wine\_data}\SpecialCharTok{$}\NormalTok{Class }\SpecialCharTok{==} \DecValTok{1}\NormalTok{, }\StringTok{"Total phenols"}\NormalTok{]}
\NormalTok{class2 }\OtherTok{\textless{}{-}}\NormalTok{ wine\_data[wine\_data}\SpecialCharTok{$}\NormalTok{Class }\SpecialCharTok{==} \DecValTok{2}\NormalTok{, }\StringTok{"Total phenols"}\NormalTok{]}
\NormalTok{class3 }\OtherTok{\textless{}{-}}\NormalTok{ wine\_data[wine\_data}\SpecialCharTok{$}\NormalTok{Class }\SpecialCharTok{==} \DecValTok{3}\NormalTok{, }\StringTok{"Total phenols"}\NormalTok{]}

\NormalTok{t\_test\_1\_2 }\OtherTok{\textless{}{-}} \FunctionTok{t.test}\NormalTok{(class1, class2)}
\NormalTok{t\_test\_1\_3 }\OtherTok{\textless{}{-}} \FunctionTok{t.test}\NormalTok{(class1, class3)}
\NormalTok{t\_test\_2\_3 }\OtherTok{\textless{}{-}} \FunctionTok{t.test}\NormalTok{(class2, class3)}

\NormalTok{t\_test\_1\_2}
\end{Highlighting}
\end{Shaded}

\begin{verbatim}

    Welch Two Sample t-test

data:  class1 and class2
t = 7.4206, df = 119.14, p-value = 1.889e-11
alternative hypothesis: true difference in means is not equal to 0
95 percent confidence interval:
 0.4261870 0.7364055
sample estimates:
mean of x mean of y 
 2.840169  2.258873 
\end{verbatim}

\begin{Shaded}
\begin{Highlighting}[]
\NormalTok{t\_test\_1\_3}
\end{Highlighting}
\end{Shaded}

\begin{verbatim}

    Welch Two Sample t-test

data:  class1 and class3
t = 17.12, df = 98.356, p-value < 2.2e-16
alternative hypothesis: true difference in means is not equal to 0
95 percent confidence interval:
 1.026801 1.296038
sample estimates:
mean of x mean of y 
 2.840169  1.678750 
\end{verbatim}

\begin{Shaded}
\begin{Highlighting}[]
\NormalTok{t\_test\_2\_3}
\end{Highlighting}
\end{Shaded}

\begin{verbatim}

    Welch Two Sample t-test

data:  class2 and class3
t = 7.0125, df = 116.91, p-value = 1.622e-10
alternative hypothesis: true difference in means is not equal to 0
95 percent confidence interval:
 0.4162855 0.7439610
sample estimates:
mean of x mean of y 
 2.258873  1.678750 
\end{verbatim}

Class1 vs Class2: The p-value = 1.889e-11, less than 0.05, meaning the
difference in the level of Total phenols between class1 and class2 is
statistically significant. Class1 has a higher mean level of phenols
than class2.

Class1 vs Class3: The p-value \textless{} 2.2e-16, meaning the
difference in the level of Total phenols between class1 and class3 is
statistically significant. Class1 has a higher mean level of phenols
than class2.

Class2 vs Class3: The p-value = 1.622e-10, less than 0.05, meaning the
difference in the level of Total phenols between class1 and class2 is
statistically significant. Class2 has a higher mean level of phenols
than class3.

\subsection{Problem 2}\label{problem-2}

Please download
\href{https://dept.stat.lsa.umich.edu/~jerrick/courses/stat506_f24/data/AskAManager.csv}{this
dataset}. It is from an ongoing salary survey from AskAManager.org.
We're going to do some data cleaning to prepare it for an analysis.

\begin{enumerate}
\def\labelenumi{\alph{enumi}.}
\tightlist
\item
  Import the data into a \texttt{data.frame} in R. As with the wine
  data, you may download the data outside of your submission, but
  importation should take place inside the problem set submission.
\end{enumerate}

\begin{Shaded}
\begin{Highlighting}[]
\NormalTok{Ask\_data }\OtherTok{\textless{}{-}} \FunctionTok{read\_csv}\NormalTok{(}\StringTok{"\textasciitilde{}/Downloads/AskAManager.csv"}\NormalTok{)}
\end{Highlighting}
\end{Shaded}

\begin{verbatim}
New names:
Rows: 28062 Columns: 19
-- Column specification
-------------------------------------------------------- Delimiter: "," chr
(16): Timestamp, How.old.are.you., What.industry.do.you.work.in., Job.ti... dbl
(3): ...1, What.is.your.annual.salary...You.ll.indicate.the.currency.in...
i Use `spec()` to retrieve the full column specification for this data. i
Specify the column types or set `show_col_types = FALSE` to quiet this message.
* `` -> `...1`
\end{verbatim}

\begin{enumerate}
\def\labelenumi{\alph{enumi}.}
\setcounter{enumi}{1}
\tightlist
\item
  Clean up the variable names. Simplify them.
\end{enumerate}

\begin{Shaded}
\begin{Highlighting}[]
\FunctionTok{colnames}\NormalTok{(Ask\_data) }\OtherTok{\textless{}{-}} \FunctionTok{c}\NormalTok{(}\StringTok{"Number"}\NormalTok{,}\StringTok{"Time"}\NormalTok{,}\StringTok{"Age"}\NormalTok{,}\StringTok{"Working Industry"}\NormalTok{,}\StringTok{"Job title"}\NormalTok{,}\StringTok{"Job Additional Context"}\NormalTok{,}\StringTok{"Annual Salary"}\NormalTok{,}\StringTok{"Additional Compensation"}\NormalTok{,}\StringTok{"Currency"}\NormalTok{,}\StringTok{"Other Currency"}\NormalTok{,}\StringTok{"Income Additional Context"}\NormalTok{,}\StringTok{"Country"}\NormalTok{,}\StringTok{"US Working Location"}\NormalTok{,}\StringTok{"City"}\NormalTok{,}\StringTok{"Overall Work Experience"}\NormalTok{,}\StringTok{"Field Work Experience"}\NormalTok{,}\StringTok{"Education Level"}\NormalTok{,}\StringTok{"Gender"}\NormalTok{,}\StringTok{"Race"}\NormalTok{)}
\NormalTok{valid\_columns }\OtherTok{\textless{}{-}} \FunctionTok{names}\NormalTok{(Ask\_data) }\SpecialCharTok{!=} \StringTok{""} \SpecialCharTok{\&} \SpecialCharTok{!}\FunctionTok{is.na}\NormalTok{(}\FunctionTok{names}\NormalTok{(Ask\_data))}
\NormalTok{Ask\_data\_valid }\OtherTok{\textless{}{-}}\NormalTok{ Ask\_data[, valid\_columns]}
\end{Highlighting}
\end{Shaded}

\begin{enumerate}
\def\labelenumi{\alph{enumi}.}
\setcounter{enumi}{2}
\tightlist
\item
  Restrict the data to those being paid in US dollars (USD). Show that
  it worked by confirming the number of observations before and after
  restricting the data.
\end{enumerate}

\begin{Shaded}
\begin{Highlighting}[]
\NormalTok{Ask\_data\_usd }\OtherTok{\textless{}{-}}\NormalTok{ Ask\_data\_valid }\SpecialCharTok{\%\textgreater{}\%} \FunctionTok{filter}\NormalTok{(Currency }\SpecialCharTok{==} \StringTok{"USD"}\NormalTok{)}
\FunctionTok{nrow}\NormalTok{(Ask\_data\_usd)}
\end{Highlighting}
\end{Shaded}

\begin{verbatim}
[1] 23374
\end{verbatim}

\begin{Shaded}
\begin{Highlighting}[]
\FunctionTok{nrow}\NormalTok{(Ask\_data)}
\end{Highlighting}
\end{Shaded}

\begin{verbatim}
[1] 28062
\end{verbatim}

\begin{enumerate}
\def\labelenumi{\alph{enumi}.}
\setcounter{enumi}{3}
\tightlist
\item
  Assume no one starts working before age 18. Eliminate any rows for
  which their age, years of experience in their field, and years of
  experience total are impossible. Again, confirm the number of
  observations. (Hint: Making these variables \texttt{factor} may make
  your life easier.)
\end{enumerate}

\begin{Shaded}
\begin{Highlighting}[]
\NormalTok{Ask\_data\_valid }\SpecialCharTok{\%\textgreater{}\%} \FunctionTok{select}\NormalTok{(Age) }\SpecialCharTok{\%\textgreater{}\%} \FunctionTok{distinct}\NormalTok{()}
\end{Highlighting}
\end{Shaded}

\begin{verbatim}
# A tibble: 7 x 1
  Age       
  <chr>     
1 25-34     
2 45-54     
3 35-44     
4 18-24     
5 65 or over
6 55-64     
7 under 18  
\end{verbatim}

\begin{Shaded}
\begin{Highlighting}[]
\NormalTok{Ask\_data\_valid }\SpecialCharTok{\%\textgreater{}\%} \FunctionTok{select}\NormalTok{(}\StringTok{\textasciigrave{}}\AttributeTok{Overall Work Experience}\StringTok{\textasciigrave{}}\NormalTok{) }\SpecialCharTok{\%\textgreater{}\%} \FunctionTok{distinct}\NormalTok{()}
\end{Highlighting}
\end{Shaded}

\begin{verbatim}
# A tibble: 8 x 1
  `Overall Work Experience`
  <chr>                    
1 5-7 years                
2 8 - 10 years             
3 2 - 4 years              
4 21 - 30 years            
5 11 - 20 years            
6 1 year or less           
7 41 years or more         
8 31 - 40 years            
\end{verbatim}

\begin{Shaded}
\begin{Highlighting}[]
\NormalTok{Ask\_data\_valid }\SpecialCharTok{\%\textgreater{}\%} \FunctionTok{select}\NormalTok{(}\StringTok{\textasciigrave{}}\AttributeTok{Field Work Experience}\StringTok{\textasciigrave{}}\NormalTok{) }\SpecialCharTok{\%\textgreater{}\%} \FunctionTok{distinct}\NormalTok{()}
\end{Highlighting}
\end{Shaded}

\begin{verbatim}
# A tibble: 8 x 1
  `Field Work Experience`
  <chr>                  
1 5-7 years              
2 2 - 4 years            
3 21 - 30 years          
4 11 - 20 years          
5 1 year or less         
6 8 - 10 years           
7 31 - 40 years          
8 41 years or more       
\end{verbatim}

\begin{Shaded}
\begin{Highlighting}[]
\NormalTok{experience\_levels }\OtherTok{\textless{}{-}} \FunctionTok{c}\NormalTok{(}\StringTok{"1 year or less"}\NormalTok{, }\StringTok{"2 {-} 4 years"}\NormalTok{, }\StringTok{"5{-}7 years"}\NormalTok{, }\StringTok{"8 {-} 10 years"}\NormalTok{, }\StringTok{"11 {-} 20 years"}\NormalTok{, }\StringTok{"21 {-} 30 years"}\NormalTok{, }\StringTok{"31 {-} 40 years"}\NormalTok{, }\StringTok{"41 years or more"}\NormalTok{)}

\NormalTok{Ask\_data\_18 }\OtherTok{\textless{}{-}}\NormalTok{ Ask\_data\_valid }\SpecialCharTok{\%\textgreater{}\%} \FunctionTok{mutate}\NormalTok{(}\AttributeTok{Age =} \FunctionTok{factor}\NormalTok{(Age),}\StringTok{\textasciigrave{}}\AttributeTok{Overall Work Experience}\StringTok{\textasciigrave{}} \OtherTok{=} \FunctionTok{factor}\NormalTok{(}\StringTok{\textasciigrave{}}\AttributeTok{Overall Work Experience}\StringTok{\textasciigrave{}}\NormalTok{, }\AttributeTok{levels =}\NormalTok{ experience\_levels, }\AttributeTok{ordered =} \ConstantTok{TRUE}\NormalTok{),}\StringTok{\textasciigrave{}}\AttributeTok{Field Work Experience}\StringTok{\textasciigrave{}} \OtherTok{=} \FunctionTok{factor}\NormalTok{(}\StringTok{\textasciigrave{}}\AttributeTok{Field Work Experience}\StringTok{\textasciigrave{}}\NormalTok{, }\AttributeTok{levels =}\NormalTok{ experience\_levels, }\AttributeTok{ordered =} \ConstantTok{TRUE}\NormalTok{)) }\SpecialCharTok{\%\textgreater{}\%} \FunctionTok{filter}\NormalTok{(Age }\SpecialCharTok{!=} \StringTok{"under 18"}\NormalTok{)}

\NormalTok{Ask\_data\_cleaned }\OtherTok{\textless{}{-}}\NormalTok{ Ask\_data\_18 }\SpecialCharTok{\%\textgreater{}\%}
  \FunctionTok{filter}\NormalTok{(}\StringTok{\textasciigrave{}}\AttributeTok{Field Work Experience}\StringTok{\textasciigrave{}} \SpecialCharTok{\textless{}=} \StringTok{\textasciigrave{}}\AttributeTok{Overall Work Experience}\StringTok{\textasciigrave{}}\NormalTok{)}
\end{Highlighting}
\end{Shaded}

\begin{enumerate}
\def\labelenumi{\alph{enumi}.}
\setcounter{enumi}{4}
\tightlist
\item
  A lot of the incomes are likely false. Eliminate any rows with
  extremely low or extremely high salaries. I'll leave the decision of
  what thresholds to use up to you; you could choose to eliminate only
  impossible values, or you could restrict the sample to eliminate the
  extreme values even if they are realistic (e.g.~removing the
  billionaires or the folks making \textless{} \$1,000 per year). You
  must justify your choice, along with either a cited source or an
  exploration the data, or some combination.
\end{enumerate}

\begin{Shaded}
\begin{Highlighting}[]
\NormalTok{low\_income\_threshold }\OtherTok{\textless{}{-}} \DecValTok{1000}
\NormalTok{high\_income\_threshold }\OtherTok{\textless{}{-}} \DecValTok{1000000}
\NormalTok{filtered\_data }\OtherTok{\textless{}{-}}\NormalTok{ Ask\_data\_cleaned }\SpecialCharTok{\%\textgreater{}\%}
  \FunctionTok{filter}\NormalTok{(}\StringTok{\textasciigrave{}}\AttributeTok{Annual Salary}\StringTok{\textasciigrave{}} \SpecialCharTok{\textgreater{}=}\NormalTok{ low\_income\_threshold }\SpecialCharTok{\&} \StringTok{\textasciigrave{}}\AttributeTok{Annual Salary}\StringTok{\textasciigrave{}} \SpecialCharTok{\textless{}=}\NormalTok{ high\_income\_threshold)}
\NormalTok{filtered\_data}
\end{Highlighting}
\end{Shaded}

\begin{verbatim}
# A tibble: 27,614 x 19
   Number Time       Age   `Working Industry` `Job title` Job Additional Conte~1
    <dbl> <chr>      <fct> <chr>              <chr>       <chr>                 
 1      1 4/27/2021~ 25-34 Education (Higher~ Research a~ <NA>                  
 2      2 4/27/2021~ 25-34 Computing or Tech  Change & I~ <NA>                  
 3      3 4/27/2021~ 25-34 Accounting, Banki~ Marketing ~ <NA>                  
 4      4 4/27/2021~ 25-34 Nonprofits         Program Ma~ <NA>                  
 5      5 4/27/2021~ 25-34 Accounting, Banki~ Accounting~ <NA>                  
 6      6 4/27/2021~ 25-34 Education (Higher~ Scholarly ~ <NA>                  
 7      7 4/27/2021~ 25-34 Publishing         Publishing~ <NA>                  
 8      8 4/27/2021~ 25-34 Education (Primar~ Librarian   High school, FT       
 9      9 4/27/2021~ 45-54 Computing or Tech  Systems An~ Data developer/ETL De~
10     10 4/27/2021~ 35-44 Accounting, Banki~ Senior Acc~ <NA>                  
# i 27,604 more rows
# i abbreviated name: 1: `Job Additional Context`
# i 13 more variables: `Annual Salary` <dbl>, `Additional Compensation` <dbl>,
#   Currency <chr>, `Other Currency` <chr>, `Income Additional Context` <chr>,
#   Country <chr>, `US Working Location` <chr>, City <chr>,
#   `Overall Work Experience` <ord>, `Field Work Experience` <ord>,
#   `Education Level` <chr>, Gender <chr>, Race <chr>
\end{verbatim}

The sample size is 27614 now.

\begin{enumerate}
\def\labelenumi{\alph{enumi}.}
\setcounter{enumi}{5}
\tightlist
\item
  (Optional) If you want to see this analysis through for \emph{no
  credit}, answer the research question of whether there is a
  statistical association between education and salary, controlling for
  years of experience.
\end{enumerate}

\subsection{Problem 3}\label{problem-3}

Palindromic numbers are integers that are equal to the reverse of their
digits. For example, 59195 is palindromic, whereas 59159 is not.

\begin{enumerate}
\def\labelenumi{\alph{enumi}.}
\item
  Write function \texttt{isPalindromic} that checks if a given positive
  integer is a palindrome. Be sure to provide a reasonable error on an
  invalid input. Be sure to document your function (see instructions
  above).

  \begin{itemize}
  \item
    Input: A positive integer
  \item
    Output: A list with two elements:

    \begin{itemize}
    \item
      \texttt{isPalindromic}: A logical value indicating if the input is
      palindromic.
    \item
      \texttt{reversed}: The input with its digits reversed.
    \end{itemize}
  \end{itemize}

  E.g.,

\begin{verbatim}
> isPalindromic(728827)
$isPalindromic
[1] TRUE

$reversed
[1] 728827

> isPalindromic(39951)
$isPalindromic
[1] FALSE

$reversed
[1] 15993
\end{verbatim}
\end{enumerate}

\begin{Shaded}
\begin{Highlighting}[]
\NormalTok{isPalindromic }\OtherTok{\textless{}{-}} \ControlFlowTok{function}\NormalTok{(n) \{}
  \ControlFlowTok{if}\NormalTok{ (}\SpecialCharTok{!}\FunctionTok{is.numeric}\NormalTok{(n) }\SpecialCharTok{||}\NormalTok{ n }\SpecialCharTok{\textless{}=} \DecValTok{0} \SpecialCharTok{||}\NormalTok{ n }\SpecialCharTok{!=} \FunctionTok{as.integer}\NormalTok{(n)) \{}
    \FunctionTok{stop}\NormalTok{(}\StringTok{"Error: Input must be a positive integer."}\NormalTok{)}
\NormalTok{  \}}
  
\NormalTok{  n\_str }\OtherTok{\textless{}{-}} \FunctionTok{as.character}\NormalTok{(n)}
  
\NormalTok{  reversed\_str }\OtherTok{\textless{}{-}} \FunctionTok{paste}\NormalTok{(}\FunctionTok{rev}\NormalTok{(}\FunctionTok{strsplit}\NormalTok{(n\_str, }\ConstantTok{NULL}\NormalTok{)[[}\DecValTok{1}\NormalTok{]]), }\AttributeTok{collapse =} \StringTok{""}\NormalTok{)}
  
\NormalTok{  is\_palindromic }\OtherTok{\textless{}{-}}\NormalTok{ n\_str }\SpecialCharTok{==}\NormalTok{ reversed\_str}
  
  \FunctionTok{return}\NormalTok{(}\FunctionTok{list}\NormalTok{(}\AttributeTok{isPalindromic =}\NormalTok{ is\_palindromic, }\AttributeTok{reversed =} \FunctionTok{as.integer}\NormalTok{(reversed\_str)))}
\NormalTok{\}}

\FunctionTok{print}\NormalTok{(}\FunctionTok{isPalindromic}\NormalTok{(}\DecValTok{728827}\NormalTok{))}
\end{Highlighting}
\end{Shaded}

\begin{verbatim}
$isPalindromic
[1] TRUE

$reversed
[1] 728827
\end{verbatim}

\begin{Shaded}
\begin{Highlighting}[]
\FunctionTok{print}\NormalTok{(}\FunctionTok{isPalindromic}\NormalTok{(}\DecValTok{39951}\NormalTok{))}
\end{Highlighting}
\end{Shaded}

\begin{verbatim}
$isPalindromic
[1] FALSE

$reversed
[1] 15993
\end{verbatim}

\begin{enumerate}
\def\labelenumi{\alph{enumi}.}
\setcounter{enumi}{1}
\item
  Create a function \texttt{nextPalindrome} that finds the next
  palindromic number strictly greater than the input. Be sure to provide
  a reasonable error on an invalid input.

  \begin{itemize}
  \item
    Input: A positive integer
  \item
    Output: A vector of length 1 with the next palindromic number
    greater than the input
  \end{itemize}

  E.g.,

\begin{verbatim}
> nextPalindrome(7152)
[1] 7227

> nextPalindrome(765431537)
[1] 765434567
\end{verbatim}
\end{enumerate}

\begin{Shaded}
\begin{Highlighting}[]
\NormalTok{isPalindromic }\OtherTok{\textless{}{-}} \ControlFlowTok{function}\NormalTok{(n) \{}
\NormalTok{  n\_str }\OtherTok{\textless{}{-}} \FunctionTok{as.character}\NormalTok{(n)}
\NormalTok{  reversed\_str }\OtherTok{\textless{}{-}} \FunctionTok{paste}\NormalTok{(}\FunctionTok{rev}\NormalTok{(}\FunctionTok{strsplit}\NormalTok{(n\_str, }\ConstantTok{NULL}\NormalTok{)[[}\DecValTok{1}\NormalTok{]]), }\AttributeTok{collapse =} \StringTok{""}\NormalTok{)}
  \FunctionTok{return}\NormalTok{(n\_str }\SpecialCharTok{==}\NormalTok{ reversed\_str)}
\NormalTok{\}}

\NormalTok{nextPalindrome }\OtherTok{\textless{}{-}} \ControlFlowTok{function}\NormalTok{(n) \{}
  \ControlFlowTok{if}\NormalTok{ (}\SpecialCharTok{!}\FunctionTok{is.numeric}\NormalTok{(n) }\SpecialCharTok{||}\NormalTok{ n }\SpecialCharTok{\textless{}=} \DecValTok{0} \SpecialCharTok{||}\NormalTok{ n }\SpecialCharTok{!=} \FunctionTok{as.integer}\NormalTok{(n)) \{}
    \FunctionTok{stop}\NormalTok{(}\StringTok{"Error: Input must be a positive integer."}\NormalTok{)}
\NormalTok{  \}}
  
\NormalTok{  next\_n }\OtherTok{\textless{}{-}}\NormalTok{ n }\SpecialCharTok{+} \DecValTok{1}
  \ControlFlowTok{while}\NormalTok{ (}\SpecialCharTok{!}\FunctionTok{isPalindromic}\NormalTok{(next\_n)) \{}
\NormalTok{    next\_n }\OtherTok{\textless{}{-}}\NormalTok{ next\_n }\SpecialCharTok{+} \DecValTok{1}
\NormalTok{  \}}
  
  \FunctionTok{return}\NormalTok{(next\_n)}
\NormalTok{\}}

\FunctionTok{print}\NormalTok{(}\FunctionTok{nextPalindrome}\NormalTok{(}\DecValTok{7152}\NormalTok{))}
\end{Highlighting}
\end{Shaded}

\begin{verbatim}
[1] 7227
\end{verbatim}

\begin{Shaded}
\begin{Highlighting}[]
\FunctionTok{print}\NormalTok{(}\FunctionTok{nextPalindrome}\NormalTok{(}\DecValTok{765431537}\NormalTok{))}
\end{Highlighting}
\end{Shaded}

\begin{verbatim}
[1] 765434567
\end{verbatim}

\begin{enumerate}
\def\labelenumi{\alph{enumi}.}
\setcounter{enumi}{2}
\item
  Use these functions to find the next palindrome for each of the
  following:

  \begin{enumerate}
  \def\labelenumii{\roman{enumii}.}
  \item
    391
  \item
    9928
  \item
    19272719
  \item
    109
  \item
    2
  \end{enumerate}
\end{enumerate}

Hints:

\begin{itemize}
\item
  While there are mathematical ways to approach this (and you can if you
  wish), string manipulation is fine.
\item
  Functions can \texttt{return} at any point, not just at the end of the
  code.
\item
  Be sure to consider what happens with numbers ending in 0.
\end{itemize}

\begin{Shaded}
\begin{Highlighting}[]
\NormalTok{numbers }\OtherTok{\textless{}{-}} \FunctionTok{c}\NormalTok{(}\DecValTok{391}\NormalTok{, }\DecValTok{9928}\NormalTok{, }\DecValTok{19272719}\NormalTok{, }\DecValTok{109}\NormalTok{, }\DecValTok{2}\NormalTok{)}

\NormalTok{next\_palindromes }\OtherTok{\textless{}{-}} \FunctionTok{sapply}\NormalTok{(numbers, nextPalindrome)}

\NormalTok{next\_palindromes}
\end{Highlighting}
\end{Shaded}

\begin{verbatim}
[1]      393     9999 19277291      111        3
\end{verbatim}



\end{document}
